\noindent
\textbf{Fila monótona}

\vspace{0.5em}
\hrule
\vspace{0.5em}

Seja $F$ uma fila de elementos do tipo $T$. A fila $F$ é dita \textbf{monótona} se, quando extraídos todos os elementos de $F$, eles formam uma sequência $x_1, x_2, \ldots, x_N$, onde $x_i$ é o elemento obtido na $i$-ésima extração, tais que a função $F : \mathbb{N} \to T$, com $f(i) = x_i$, é monótona.
A fila $F$ será \textbf{não-decrescente} se $f$ for \textbf{não-decrescente}; caso contrário, $F$ será \textbf{não-decrescente}.

\begin{itemize}
    \item Em filas monótonas é necessário manter a invariante da monotonicidade a cada inserção.
    
    \item Seja $F$ uma fila não-decrescente e $x$ um elemento a ser inserido em $F$.
    
    \item Se $F$ estiver vazia, basta inserir $x$ em $F$: o invariante estará preservado.
    
    \item Se $F$ não estiver vazia, o mesmo acontece se $x \le y$, onde $y$ é o último de $F$.
    
    \item Contudo, se $x > y$, é preciso remover $y$ antes da inserção de $x$.
    
    \item Após a remoção de $y$, é preciso confrontar $x$ com o novo elemento que ocupará a última posição até que $x$ possa ser inserido na última posição de $F$.
\end{itemize}